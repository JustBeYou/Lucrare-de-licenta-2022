\begin{abstractpage}

\begin{abstract}{romanian}
Sistemele IoT sunt heterogene, distribuite și complexe datorită interconectării unui număr mare de componente hardware și software. În lucrarea de față am argumentat că aceste caracteristici cresc dificultatea testării aspectelor funcționale și nefuncționale, testarea reprezentând un subiect de interes atât pentru cercetători cât și pentru dezvoltatorii industriali. În urma analizei literaturii de specialitate, am constat lipsa unor repere obiective de comparare și analiză a tehnicilor de testare, fiecare publicație utilizând propriile aplicații, dispozitive și metrici pentru evaluare. Evaluarea neomogenă în publicații îngreunează compararea obiectivă a rezultatelor și reproducerea acestora atunci când condițiile originale nu sunt disponibile. 

Pentru a depăși acest obstacol, am construit o suită open-source de aplicații care să modeleze cât mai bine caracteristicile unui sistem IoT real, suita mimând o rețea a unei locuințe inteligente. Pentru a asigura caracterul heterogen, dar și comunicarea complexă, aplicațiile au fost construite de multiple entități independente (echipe de studenți, autorul prezentei lucrări, dezvoltatori independenți) și apoi integrate în fluxuri de automatizare care implică angrenarea mai multor aplicații. Suita conține o serie de defecte (bugs), naturale sau injectate artificial, din clase variate. Am prezentat clasificarea acestora atât raportat la sisteme de categorisire standard precum Common Weakness Enumeration (CWE), dar și folosind o ierarhie proprie bazată pe nivelul de interconectare la care se manifestă defectul.

Utilizând suita construită am efectuat experimente folosind multiple tehnici de testare și analiză. Am comparat o abordare proprie de testare funcțională manuală inspirată de principiile Behavior-driven development (BDD) cu framework-ul PatrIoT, fuzzere precum RESTler și afl++, dar și metode de analiză formală și statică. În urma analizei am determinat punctele tari și slabe ale fiecări abordări și contextul cel mai potrivit pentru utilizare. Concluzionăm că realizarea suitei de aplicații este relevantă pentru analiza și îmbunătățirea practicilor de testare existente și că este necesară încurajarea creării de medii de test publice.
\end{abstract}

\newpage

\begin{abstract}{english}
TODO: Trebuie sa redactez rezumatul in engleza

Lorem ipsum dolor sit amet, consectetur adipiscing elit. Fusce vitae eros sit amet sem ornare varius. Duis eget felis eget risus posuere luctus. Integer odio metus, eleifend at nunc vitae, rutrum fermentum leo. Quisque rutrum vitae risus nec porta. Nunc eu orci euismod, ornare risus at, accumsan augue. Ut tincidunt pharetra convallis. Maecenas ut pretium ex. Morbi tellus dui, viverra quis augue at, tincidunt hendrerit orci. Lorem ipsum dolor sit amet, consectetur adipiscing elit. Aliquam quis sollicitudin nunc. Sed sollicitudin purus dapibus mi fringilla, nec tincidunt nunc eleifend. Nam ut molestie erat. Integer eros dolor, viverra quis massa at, auctor.

Lorem ipsum dolor sit amet, consectetur adipiscing elit. Fusce vitae eros sit amet sem ornare varius. Duis eget felis eget risus posuere luctus. Integer odio metus, eleifend at nunc vitae, rutrum fermentum leo. Quisque rutrum vitae risus nec porta. Nunc eu orci euismod, ornare risus at, accumsan augue. Ut tincidunt pharetra convallis. Maecenas ut pretium ex. Morbi tellus dui, viverra quis augue at, tincidunt hendrerit orci. Lorem ipsum dolor sit amet, consectetur adipiscing elit. Aliquam quis sollicitudin nunc. Sed sollicitudin purus dapibus mi fringilla, nec tincidunt nunc eleifend. Nam ut molestie erat. Integer eros dolor, viverra quis massa at, auctor.

Lorem ipsum dolor sit amet, consectetur adipiscing elit. Fusce vitae eros sit amet sem ornare varius. Duis eget felis eget risus posuere luctus. Integer odio metus, eleifend at nunc vitae, rutrum fermentum leo. Quisque rutrum vitae risus nec porta. Nunc eu orci euismod, ornare risus at, accumsan augue. Ut tincidunt pharetra convallis. Maecenas ut pretium ex. Morbi tellus dui, viverra quis augue at, tincidunt hendrerit orci. Lorem ipsum dolor sit amet, consectetur adipiscing elit. Aliquam quis sollicitudin nunc. Sed sollicitudin purus dapibus mi fringilla, nec tincidunt nunc eleifend. Nam ut molestie erat. Integer eros dolor, viverra quis massa at, auctor.
\end{abstract}

\end{abstractpage}