\begin{abstractpage}

\begin{abstract}{romanian}

Sistemele IoT sunt eterogene, distribuite și complexe datorită interconectării unui număr mare de componente hardware și software. În lucrarea de față, am argumentat că aceste caracteristici cresc dificultatea testării aspectelor funcționale și nefuncționale, testarea reprezentând un subiect de interes atât pentru cercetători, cât și pentru dezvoltatorii industriali. 

În urma analizei literaturii de specialitate, am constatat lipsa unor repere obiective de comparare și analiză a tehnicilor de testare, fiecare publicație utilizând propriile aplicații, dispozitive și metrici pentru evaluare. Astfel, reproducerea și compararea rezultatelor este dificilă.

Pentru a depăși acest obstacol, am construit o suită open-source de aplicații, care să modeleze cât mai bine caracteristicile unui sistem IoT real, suita mimând o rețea a unei locuințe inteligente. Pentru a asigura caracterul eterogen, dar și comunicarea complexă, aplicațiile au fost construite de multiple entități independente (echipe de studenți, autorul prezentei lucrări, dezvoltatori independenți) și apoi integrate în fluxuri de automatizare care implică angrenarea mai multor aplicații. Suita conține o serie de defecte (bugs) naturale sau injectate artificial, din clase variate. Am prezentat clasificarea acestora atât raportat la sisteme de categorisire standard, precum Common Weakness Enumeration (CWE), dar și folosind o ierarhie proprie bazată pe nivelul de interconectare la care se manifestă defectul.

Utilizând suita construită, am efectuat experimente demonstrative, folosind diferite tehnici de testare, cum ar fi testare funcțională manuală, analiză statică a codului sursă, \textit{fuzzing} și verificare formală. În urma experimentelor, am tras concluzia că realizarea suitei de aplicații este relevantă pentru analiza și îmbunătățirea practicilor de testare existente și că este necesară încurajarea creării de medii de test publice.

O parte din rezultatele prezentate în această lucrare au fost deja publicate în articolul realizat de \fullcite{Cristea2022}.

Îi mulțumesc domnului profesor Alin Ștefănescu pentru sprijinul fără de care realizarea acestei lucrări nu ar fi fost posibilă. Le mulțumesc, de asemenea, domnului profesor Ciprian Păduraru și doctorandului Rareș Cristea pentru sfaturile oferite și colaborarea din timpul activității mele de cercetare.
\end{abstract}

\newpage

\begin{abstract}{english}
IoT systems are heterogeneous, distributed, and complex due to a large number of interconnected software and hardware components in their composition.  In this paper, we argue that given these characteristics, functional and non-functional testing of IoT systems is considerably more difficult, testing being of high interest for researchers and practitioners. 

Analyzing the related work, we found a lack of objective benchmarks for comparing testing techniques, because, usually, each published paper describes a custom set of applications, devices, and evaluation metrics. Thus, reproducing and comparing results is difficult.

To overcome the presented challenges, we built an open-source set of applications, which should mimic the characteristics of a real IoT system, the set being a simulation of a smart home.  To assure that we have a heterogeneous system in which complex communication is involved, the applications were built by multiple independent parties (students, the author of this paper, and other developers) and then integrated using automation flows. The application set contains multiple, real and injected, bugs of different types. We offer a custom classification based on the degree of interconnection.

Using the described set, we ran a series of experiments using testing techniques such as manual functional testing, static analysis, fuzzing, and formal verification. We concluded that the set of applications we built is relevant for advancing the state-of-the-art of IoT testing techniques.

Part of this research was already published in the paper written by \fullcite{Cristea2022}.

I would like to thank Professor Alin Ștefănescu for their support, without which this work would not have been possible. I also thank Professor Ciprian Păduraru and PhD student Rareș Cristea for their advice and collaboration during my research.

\end{abstract}

\end{abstractpage}