\chapter{Introducere}

\acrlong{iot} este unul din subiectele de cel mai mare interes în sfera tehnologiei,
alături de inteligența artificială, tehnologia blockchain și realitatea virtuală. 
Definiția exactă a acestui termen variează semnificativ atât în lucrările
științifice cât și în presă sau publicațiile companiilor din domeniul tehnologiei informației sau domenii adiacente. Prima dată utilizat de Kevion Ashton într-o prezentare ținută pentru Procter \& Gamble în anul 1999 \cite{ashton_2009}, acesta se referea la dispozitivele care cu ajutorul senzorilor dau capacitatea computerelor de a "vedea", "auzi" și "simți" mediul înconjurător. Astăzi, atunci când vorbim de \acrshort{iot} înglobăm o gamă largă de concepte și dispozitive: rețele wireless, electrocasnice inteligente, automatizări rezidențiale sau industriale, vehicule autonome, toate se pot încadra sub eticheta \acrshort{iot}.

Probabil cea mai răspândită și populară aplicare a sistemelor \acrshort{iot} sunt locuințele inteligente. Datorită interesului crescut pentru eficientizarea consumului de energie, reducerea emisiilor de carbon, dar și al beneficiilor promise pentru calitatea vieții, locuințele și orașele inteligente interconectate prin internet au devenit un vis tehnologic atât al cetățenilor cât și al companiilor sau guvernelor. De la iluminare cu senzori de mișcare până la asistenți virtuali care ne învață preferințele legate de muzică sau temperatură, suntem tot mai înconjurați de \textit{lucruri} (\textit{things}) inteligente, interconectate, ce procesează cantități enorme de date despre mediul nostru, dar și despre noi. 

Deși este o nișă relativ tânără, rețelele de dispozitive inteligente își fac loc în tot mai multe industrii și domenii de activitate cu o istorie lungă. În fabricile moderne se utilizează rețele complexe de senzori, roboți și dispozitive de coordonare pentru a facilita linii de producție. În agricultură atât monitorizarea cât și îngrijirea culturilor se poate realiza folosind senzori și drone conectate la internet. În sectorul public există inițiative de digitalizare a infrastructurii și comunicării dintre instituții și cetățeni, scopul final fiind creearea de orașe cu adevărat inteligente. 

Considerând importanța contextului prezentat, lucrarea de față își propune să analizeze metodele existente de testare a sistemelor \acrshort{iot}, în principal aspectele funcționale, dar cu un interes aparte pentru una din proprietățile nefuncționale, și anume securitatea. Pe lângă analiza teoretică a literaturii și practicilor existente, voi include rezultate experimentale obținute atât prin reproducerea experimentelor din literatură, cât și ale abordărilor proprii.

Capitolul curent conține în secțiunile ce urmează o analiză detaliată a motivației și interesului pentru îmbunătățirea tehnicilor de testare a sistemelor \acrshort{iot}, un scurt istoric al domeniului, un sumar al contribuțiilor prezentate, iar în final o prezentare a structurii și conținutului prezentei lucrări.

\section{Interesul și motivația pentru temă}

\section{Cronologie a Internet of Things}

\section{Structura lucrării}

\section{Alte considerații}