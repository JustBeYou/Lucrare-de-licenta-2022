\chapter{Modelarea sistemelor Internet of Things}

Pentru o analiză cât mai riguroasă este nevoie de o înțelegere profundă a sistemelor testate. În acest capitol vom oferi o imagine de ansamblu asupra rețelelor \acrshort{iot} ilustrând caracteristicile acestora cu exemple reale din diferite industrii. Pe baza exemplelor expuse vom discuta despre caracteristicile arhitecturale ale rețelelor, diferitele părți componente, topologiile comune și care este parcursul fluxurilor de date. Deoarece nu există o viziune unificată a modului în care ar trebui descrise arhitecturile acestor rețele particulare, vor fi expuse mai multe alternative și voi argumenta care este cea mai potrivită pentru a fi utilizată. 

Arhitecturile specifice \acrshort{iot} impun anumite constrângeri metodelor de comunicație, așa că vom explora care sunt cele mai comune protocoale de comunicație utilizate, care sunt particularitățile lor și care este motivația din spatele utilizării acestora. Nu ne vom rezuma doar la protocoalele dintr-un singur nivel al modelului \acrfull{osi}, ci vom urmări o privire de ansamblu, un interes special va fi acordat protocoalelor precum \textit{Zig-Bee} care este proiectat special pentru \acrshort{iot} și se întinde pe mai multe nivele \acrshort{osi}.

Canalele de comunicație au nevoie de părți comunicante, așa că vom continua prin a analiza dispozitivele și caracteristicile acestora utilizate în sistemele \acrshort{iot}, un punct important în această discuție fiind despre limitările hardware impuse de eficiența costurilor și cum acest aspect afectează capabilitățile de securitate. Deoarece software-ul este strâns legat de hardware în acest domeniu discuția va fi extinsă și asupra interacțiunii dintre cele două, în special despre modul în care dezvoltarea sau testarea uneia dintre componente o afectează pe cealaltă.

În finalul capitolului vom trece din planul tehnic și tehnologic în cel teoretic și vom modela formal și riguros rețelele \acrshort{iot} folosindu-ne de teoria grafurilor și teoria automatelor finite. Vor fi descrise atât topologiile și fluxurile de date fizice, cât și fluxurile la nivel logic. Ne propunem să obținem o viziune clară asupra proprietăților acestor sisteme pentru a putea exprima mai ușor aspectele testate și pentru a deschide calea spre metode de testare formală cum ar fi cele bazate pe \acrfull{smt}.

Un material suplimentar va fi reprezentat de un studiu de caz asupra sistemelor din locuințele inteligente unde vom identica aspectele discutate anterior și vom ilustra prin exemple aplicarea modelelor teoretice pentru descrierea unui astfel de sistem. Locuințele inteligente vor reprezenta exemplul principal în capitolele ce urmează atunci când vom aduce în prim plan tehnicile de testare.

\section{Arhitectură}

Așa cum ne spun \cite{Khodadadi2016}, componentele fundamentale și arhicunoscute ale sistemelor \acrshort{iot} sunt dispozitivele cu capabilități senzoriale, apelarea de servicii de la distanță, comunicarea peste rețea și procesarea de evenimente într-un context specific. Aceste părți componente puse împreună creează sisteme cu caracter puternic heterogen așa cum am menționat și anterior. Prin urmare, asigurarea interconectivității și interoperabilității devine un obiectiv greu de atins în lipsa unor procedee corespunzătoare de abstractizare. Aceste procedee de abstractizare se regăsesc în modelele arhitecturale propuse pentru sistemele \acrshort{iot}, modele pe care le vom analiza în paragrafele următoare, dar nu înainte de a arunca o scurtă privire asupra câtorva exemple practice.

Pentru a înțelege mai bine nevoia de abstractizare, să analizăm câteva exemple din lumea reală. Ne putem imagina o locuință inteligentă în care avem instalați senzori multiplii pentru temperatură, luminozitate, mișcare, etc. Aceștia transmit date către un computer central care le agregă și le pune la dispoziție utilizatorilor locuinței. De asemenea, utilizatorii își pot stabili pe baza datelor colectate o serie de automatizări, cum ar fi mișcarea draperiilor, reglarea temperaturii, încuierea ușii, etc. Toate facilitățile pot fi acționate de la distanță prin intermediul internetului. În plus, locuința este conectată la sistemul inteligent al municipalității pentru a comunica informații despre consumul de energie, apă sau alte resurse. În cadrul sistemului municipal, scenariul se repetă. Regăsim o serie de senzori, centre de comandă și dispozitive de acțiune. Este evident cât de rapid crește complexitatea chiar și într-un caz relativ restrâns.

Un alt exemplu poate fi reprezentat de o linie de producție industrială complet automatizată. Multiplii senzori colectează date despre funcționarea aparaturii. Unul sau mai multe centre de comandă interpretează datele pentru a orchestra brațele robotice din cadrul liniei de producție. Angajații respectivei fabrici pot interacționa cu sistemele acesteia pentru monitorizare și control manual atunci când este nevoie. Toate datele conectate sunt trimise către \textit{cloud}-ul companiei pentru analiza și optimizarea proceselor de producție. Acest gen de sisteme reprezintă o nișă specifică, anume \acrfull{iiot}. 

% TODO: sa bag niste diagrame misto cu un sistem smart home si cu unul industrial, poate chiar o diagrama din Node-RED

Modelele arhitecturale au evoluat în timp și continuă să evolueze, așa că le vom analiza în ordinea complexității lor.

\subsection*{\textit{3-layer}}
% TODO diagram 3 layer

\cite{MiaoWu2010} spune că în ciuda lipsei unei definiții unificate pentru \acrshort{iot}, arhitectura \textit{3-layer} este larg acceptată și cunoscută. Acest fapt este bine fundamentat, deoarece găsim referințe la acest model arhitectural în multiple publicații la care ne vom referi în paragrafele următoare. Nivelele componente ale acestei arhitecturi sunt:

\begin{enumerate}
    \item Nivelul de percepție (\textit{perception layer})
    \item Nivelul rețelei (\textit{network layer})
    \item Nivelul aplicației (\textit{application layer})
\end{enumerate}

Nivelul de percepție conține senzori, cititoare de \acrshort{rfid} sau coduri de bare, sisteme \acrfull{gps} sau camere de filmat. Acesta facilitează colectarea de date despre mediul înconjurător sistemului \acrshort{iot}. Identificăm cu ușurință acest nivel în exemplul nostru anterior, senzorii termici sau de mișcare, cât și dispozitivele de măsurat consumul de energie sau apă fac parte din acesta. Acest nivel poate fi de asemenea considerat a fi colecția de \textit{things} din \acrshort{iot}.

La nivelul rețelei întâlnim infrastructura care pune la dispoziție comunicarea dintre \textit{things}. Această infrastructură este formată din \textit{router}-e, \textit{switch}-uri, centre de procesare și management, etc. De exemplu, într-o locuință inteligentă rețeaua \textit{wireless} și centrul de control reprezintă nivelul de rețea.

Procesarea datelor și executarea fluxurilor specifice se petrece la nivelul aplicației. Acest nivel reprezintă totalitatea componentelor software și hardware care facilitează serviciile dorite de utilizator. De exemplu, software-ul de automatizare prezent pe centrul de comandă dintr-o locuință inteligentă face parte din acest nivel.

\cite{Lin2017} constată ca deși această arhitectură este fundamentală și simplă, funcționalitățile și operațiunile desfașurate de sistem la nivelul rețelei și aplicației sunt diverse și complexe. În opinia autorilor este nevoie de dezvoltarea unui nivel de serviciu \textit{service layer} care să medieze interacțiunea dintre rețea și aplicație. Acest nivel adițional ar facilita o mai mare flexibilitate sistemului.

\subsection*{\textit{5-layer}}
% TODO diagrama 5 layer

Arhitectura \textit{3-layer} nu conține metode de management suficient de bune și nu ia în considerare nevoia de a modela domeniile de \textit{business} în perspectiva \cite{MiaoWu2010}. Pentru a combate aceste dificultăți, autorii propun o arhitectura cu cinci nivele, după cum urmează:

\begin{enumerate}
    \item Nivelul de percepție (\textit{perception layer})
    \item Nivelul de transport (\textit{transport layer})
    \item Nivelul de procesare (\textit{processing layer})
    \item Nivelul aplicației (\textit{application layer})
    \item Nivelul \textit{business} (\textit{business layer})
\end{enumerate}

Primele două nivele corespund cu cele din arhitectura \textit{3-layer}, îndeplinind aceleași funcții. În continuare, nivelul de procesare preia o parte din atribuțiile nivelului aplicației din arhitectura anterioară. Din acest nivel fac parte bazele de date, \textit{cloud}-ul și toate celelalte aparaturi de procesare de date pentru a facilita utilizarea acestora la nivelul aplicației. În viziunea autorilor, această separare dintre procesare și utilizare în scopuri aplicate este necesară pentru flexibilitatea și scalabilitatea \acrshort{iot}.

În final avem de a face cu nivelul de \textit{business} care ar trebui să reprezinte un nivel de management peste nivelul aplicației, fiind preocupat de modelul de \textit{business} și dezvoltarea pe termen lung. Din păcate, autorii nu pun la dispoziție o descriere detaliată, aplicarea acestui tip de arhitectură rămânând la latitudinea dezvoltatorului. O imagine mai clară asupra nivelului de \textit{business} ne-o oferă \cite{Khan2012} specificând că acesta conține modelele de \textit{business}, grafice și organigrame bazate pe datele obținute din nivelul de aplicație. Scopul acestui nivel este să producă strategii și decizii de \textit{business}, deci este un nivel de interacțiune între om și mașină.

\subsection*{\textit{Service Oriented Architecture}}
% TODO: diagramă

O arhitectură orientată pe servicii, în orginal \acrfull{soa}, propune împărțirea componentelor și funcționalităților unui sistem un unități mici și independente numite servicii. Această arhitectură nu impune un număr fix de nivele, însă trebuie sa existe cel puțin un nivel care facilitează două activități fundamentale: descoperirea serviciilor (\textit{service discovery}) și invocarea acestora (\textit{service calling}). 

\cite{Khodadadi2016} susțin că acest tip de arhitectură asigură interoperabilitatea dispozitivelor heterogene, această proprietate fiind esențială pentru sistemele \acrshort{iot}. De asemenea, oferă și un exemplu de \acrshort{soa} adăugând un nivel de servicii în clasica arhitectură \textit{3-layer}. Funcționarea independentă a serviciilor asigură că sistemul va continua să funcționeze majoritar corect, chiar dacă un număr mic de servicii au încetat să funcționeze. Acest aspect este foarte important pentru fiabilitatea sistemelor, o proprietate de mare interes. Același tip de arhitectură este descris și de \cite{Lin2017} într-un meta-studiu.

O altă perspectivă asupra unui posibil tip de \acrshort{soa} vine din partea \cite{LuTan2010} care propune o arhitectură \textit{5-layer} modificată. Pe langă nivelul rețelei, aceștia introduc un nivel de coordonare al serviciilor care are rolul de a unifica comunicarea asigurând interoperabilitatea. Din păcate, nu sunt oferite detalii suplimentare despre implementarea acestui tip de arhitectură.

\subsection*{\textit{API-oriented}}

Deoarece arhitecturile de tip \acrshort{soa} sunt complexe, iar procesele de coordonare și descoperire a serviciilor necesită un efort considerabil de dezvoltare, alternative mai simple au fost introduse. \cite{Khodadadi2016} prezintă o imagine de ansamblu a arhitecturilor orientate spre \acrfull{api}. % TODO: sa continui aici, nu mai am idei

\section{Protocoale de comunicație}

În discuția anterioară despre arhitectura sistemelor \acrshort{iot} am observat că un rol important este jucat de mecanismele de comunicare. Aceste sisteme sunt nevoite să coordoneze un număr mare de dispozitive și să transmită o cantități enorme de date într-un mod fiabil, rapid și rezistent la șoc. De aceea sunt utilizate o gamă largă de protocoale de comunicație, fiecare cu avantajele și dezavantajele sale, în încercarea de a depăși provocările impuse de natura distribuită și heterogenă a sistemelor. 

Este necesar să fim familiari cu modelul \acrfull{osi} pentru o bună înțelegere a diferențelor dintre diferitele protocoalele utilizate. Deoarece \acrshort{iot} este un domeniu volatil în care hardware-ul este în strânsă legătură cu software-ul, 


TODO: de vorbit despre protocoalele specifice

TODO: de inclus rezumatul despre Event-Driven-Systems

\section{Caracteristicile dispozitivelor}

\section{Model teoretic}

În încercarea de a crea un cadru obiectiv pentru analiza sistemelor IoT din punct de vedere al proprietăților și potențialelor defecte din acestea, vom introduce câteva noțiuni teoretice pentru descrierea acestor sisteme. În secțiunea anterioară am discutat despre comunicarea fizică a dispozitivelor, diferitele topologii pe care le-am putea întâlni și protocoalele utilizate. Vom extinde discuția spre fluxurile logice prezente în aceste rețele, deoarece deși topologiile de tip stea sunt des întâlnite, nodul central servește ca un intermediar pentru fluxuri logice între un număr mare de dispozitive, acestea neputând fi reduse la o interacțiune nod central - dispozitiv.

Vom folosi ca fundație specificația propusă de \cite{Paduraru2021}, deoarece propune o descriere formală a rețelelor IoT folosind teoria grafurilor pentru a ușura automatizarea proceselor de testare, fără a fi nevoie să se specifice detalii tehnice precum protocoalele sau software-ul utilizat. Autorii propun să vizualizăm o aplicație IoT ca o serie de perechi \textit{input}-\textit{output} care pot fi descrise de un graf orientat G în modul următor:

\begin{enumerate}
    \item orice vârf $v \in V$ reprezintă un dispozitiv fizic pe care se execută unul sau mai multe procese software
    \item o muchie orientată $e \in E$ de la un vârf $v_1$ la un vârf $v_2$ descrie un flux de date de la \textit{output}-ul lui $v_1$ spre \textit{input-ul} lui $v_2$
    \item orice vârf $v \in V$ este consumatorul datelor pentru un set de producători $V_{in}(v) = \Big{\{} v_{in} \; \Big{|} \; \exists v \in V \land (v_{in}, v) \in E \Big{\}}$
    \item orice vârf $v \in V$ este producător pentru un set de consumatori $V_{out}(v) = \Big{\{} v_{out} \; \Big{|} \; \exists v \in V \land (v, v_{out}) \in E \Big{\}}$
    \item orice vârf $v \in V$ este caracterizat de un \textit{buffer} de \textit{input} și unul de \textit{output}, aceștia fiind descriși de relația producător consumator astfel încât: $Buffer_{in}(v) = \bigcup_{v_{in} \in V_{in}(v)} Buffer_{out}(v_{in})$ și analog $Buffer_{out}(v) = \bigcup_{v_{out} \in V_{out}(v)} Buffer_{in}(v_{out})$
    \item graful $G$ este dinamic, vârfuri și muchii putând fii adăugate sau eliminate în timpul funcționării rețelei
\end{enumerate}

% TODO: de adaugat o diagrama si de explicat pe ea

Un aspect important discutat de autori este că muchiile rețelei trebuiesc privite dintr-o perspectivă probabilistă, anume ele pot lipsi în timpul funcționării sistemului. De exemplu, un senzor ar putea fi deconectat, fără energie sau să comunice date la intervale foarte lungi de timp. O viziune mai clară asupra acestui aspect o putem avea privind Figura ... Observăm că unele muchii sau vârfuri din graf sunt obligatorii pentru funcționarea sistemului, eliminarea lor făcând imposibilă o execuție coerentă. Acest set de vârfuri și muchii obligatorii sunt notate $V_r$ și $E_r$ și determină graful funcțional minimal $G_r$. % TODO de adaugat figura si de explica

Dacă considerăm orice graf $G \in G_{total}$ o stare posibilă a rețelei (ie. $G$ conține toate muchiile și vârfurile obligatorii), observăm ca avem de a face cu un automat finit. Putem descrie stările acestui automat considerând $S = \Big{\{} G  \; \Big{|} \; G \in G_{total} \land G_{r} \subseteq G \Big{\}}$ și operațiile posibile de modificare ale grafului $\Sigma = \Big{\{} \text{adăugare muchie}, \text{ștergere muchie}, \text{ștergere vârf}, \text{adăugare vârf} \Big{\}}$.

% TODO poate ar trebui sa citesc lucrurile astea de undeva?
% TODO poate ar fi interesant totusi sa extind un pic ideea? ar fi interesant pentru fuzzing

O descriere mult mai precisă a sistemului ar putea fi reprezentată de utilizarea lanțurilor Markov deoarece muchiile opționale au probabilități asociate, însă această abordare depășește scopul lucrării de față.

Descrierea propusă de \cite{Paduraru2021} se concentrează pe fluxurile fizice de date dintre dispozitive, însă un aspect important al acestor tipuri de sisteme este reprezentat de fluxurile logice de date, așa cum am văzut în discuția anterioară despre arhitecturi și protocoale, datele colectate de un senzor sau generate de un utilizator pot influența multiple părți ale sistemului, de la simplii actuatori până la sisteme complexe de analiză și decizie.

% TODO: sa vad cum as putea vorbi despre flow-uri logice

Într-o manieră similară celei de mai sus putem descrie graful $G'$

% TODO: de citat paper-ul lui Stefanescu + Paduraru despre privitul ca graf
% TODO: de vazut cum s-ar putea descrie state-ul retelei cu teorie de la automatele finite
% TODO: de vorbit despre numarul imens de stari, caracterul continuu al masuratorilor, etc
% TODO: de subliniat distinctia dinstre comunicarea logica (la nivel de flow) si comunicarea fizica (la nivel de protocol/device) si de modelat comunicarea logica

\section{Studiu de caz: locuințe inteligente}

TODO: de inclus rezumatul despre smart home thread modeling

TODO: de inclus studiul de la trendmicro