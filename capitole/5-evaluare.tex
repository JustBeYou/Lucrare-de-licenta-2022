\chapter{Evaluarea unor tehnici de testare}

Pentru a dovedi relevanța suitei de aplicații construită anterior vom desfășura o serie de experimente folosind mai multe tehnici și tehnologii adecvate pentru testarea sistemelor \acrshort{iot}. 

% TODO: de palavragit un pic despre cercetarea empirica si de ce e buna
% TODO: de vorbit despre analiza cantitativa vs analiza calitativa
% TODO: de ales niste metrici de evaluare si motivat de ce
% TODO: de rezumat tehnicile si structura capitolului

\section{Testarea funcțională}

% TODO: de vorbit la general despre testarea functionala, dar scurt

\subsection{Metodologia Behaviour-Driven Design}

% TODO: de folosit continutul din paper

\subsection{PatrIoT Framework}

% TODO: !!! trebuie sa fac eu experimentele cu PatrIoT si sa compar

\section{Testarea exploratorie (\emph{fuzzing})}

% TODO: de explicat ce inseamna testare exploratorie
% TODO: teorie fuzzing scurta, de gasit un paper de citat

\subsection{RESTler}

% TODO: de citat paper-ul de la restler si de folosit balariile din paper-ul de fuzzing de la SASHA

% TODO: de vazut ce mai zice rares si ce experimente mai face el

\subsection{}

% TODO: trebuie sa mai gasesc macar un fuzzer pe care sa-l rulez pe dataset ca sa am ce compara

\section{Testarea cu analiză formală}

% TODO: de gasit un paper despre analiza formala si modelare
% TODO: de reprodus macar un bug folosind modelarea formala si un solver gen Z3 sau un theorem prover