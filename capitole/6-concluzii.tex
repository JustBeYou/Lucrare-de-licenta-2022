\chapter{Concluzii}

Așa cum am stabilit la începutul lucrării, construirea setului de aplicații a fost fundamentată pe modelarea teoretică și tehnologică a sistemelor \acrshort{iot} prezentată cititorului, iar tehnicile de testare evaluate servesc drept exemplu pentru viitori cercetători interesați de îmbunătățirea practicilor de testare. Astfel, constatăm că obiectivele stabilite în capitolul 2 au fost atinse.

Luând în considerare limitările suitei de aplicații construită, întrevedem ca direcții viitoare de cercetare: extinderea setului la aplicații reale, din industrie, utilizarea mai multor protocoale de comunicație specifice, cum ar fi \acrshort{coap} sau ZigBee și diversificarea defectelor injectate în suită. Vom putea astfel să ne apropiem și mai mult de provocările din scenariile reale.

În urma analizei tehnicilor de testare, am observat necesitatea unei metodologii riguroase de utilizare a suitei de aplicații. În cercetările viitoare, ar putea fi conceput un ghid de metodologie. De asemenea, pentru transparentizarea rezultatelor, ar putea fi construită o platformă care să pună la dispoziție măsurători și evaluare automată pentru diferite tipuri de tehnici proiectate de cercetători. O inițiativă similară care ar putea servi drept model este \textit{Google Fuzz Bench} \footnote{Adresa platformei se află la \url{https://google.github.io/fuzzbench/}.}.

În plan teoretic, un subiect promițător pentru viitoare cercetări este verificarea formală. Modelele matematice din literatură care descriu sistemele \acrshort{iot} sunt încă în stadii incipiente, există astfel o plajă largă de îmbunătățiri ce pot fi aduse. 

Considerând argumentele prezentate în cadrul lucrării, putem conchide că suita de aplicații \acrlong{iot} construită își îndeplinește țelul, aceasta reprezentând un mediu stabil pentru analizarea, evaluarea și compararea tehnicilor de testare. 